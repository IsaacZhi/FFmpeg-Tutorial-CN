\chapter*{现在还要做什么?}
\label{ch9}
我们已经有了一个可以工作的播放器,当然,还还不够好。我们做了很多,但是还有很多可以添加的功能:
\begin{itemize}
\item 现在我们基于的 ffplay.c 已经过时了,这份指南需要一个做大的更新。如果你需要在更严肃的项目里面用 ffmpeg 库的话,请你参考最新的 ffplay.c 
\item 错误处理。我们代码中的错误处理是糟透的,多处理一些会更好。
\item 暂停。我们不能暂停电影,这是一个很有用的功能。我们可以在大结构体中使用一个内部暂停变量,当用户暂停的时候就设置它。然后我们的音频,视频和解码线程检测到它后就不再输出任何东西。我们也使用av_read_play 来支持网络。这很容易解释,但是你自己来做的话不那么容易,所以如果你想尝试的话,\textbf{把这个作为一个家庭作业}。提示,可以参考ffplay.c。
\item 支持视频硬件特性。一个参考的例子,请参考\href{http://www.inb.uni-luebeck.de/~boehme/libavcodec_update.html}{Martin的旧教程}中的Frame Grabbing 部分。

\item 按字节跳转。如果你可以按照字节而不是秒的方式来计算出跳转位置,那么对于像VOB 文件一样的有不连续时间戳的视频文件来说,定位会更加精确。
\item 丢弃帧。如果视频落后的太多,我们应当把下一帧丢弃掉而不是设置一个短的刷新时间。
\item 支持网络。现在的电影播放器还不能播放网络流媒体。
\item 支持像YUV 文件一样的原始视频流。如果我们的播放器支持的话,因为我们不能猜测出时基和大小,我们应该加入一些参数来进行相应的设置。
\item 全屏。
\item 各种选项,例如:不同图像格式;参考ffplay.c 中的命令开关。
\item 其它事情,例如:在结构体中的音频缓冲区应该对齐。
\end{itemize}

我们目前仅仅涵盖了ffmpeg中的一小部分,如果你想了解关于ffmpeg 更多的事情,下一步应该学习的就是如何来\emph{编码}多媒体。一个好的入手点是在 ffmpeg 中的output_example.c 文件。我可以为它写另外一个教程,但是我没有足够的时间来做。

更新:自从本文档已经很久没有更新了,现实世界的视频软件已经更加成熟。这个教程仅仅要求一些简单的API更新,基本概念更新比较少。很多更新是简化了代码。我在这里更新代码,但是 ffplay 仍然比这个玩具应用效率高。说实话,这它作为一个真正的播放器还不太够。如果你想改进这个教程,看 ffplay 找出我们缺少了什么。我猜更多的是利用视频硬件,不过也许是我少做了些什么。也许 ffplay 做了什么我没有看到的大幅度的改进。

我仍然非常骄傲这个教程帮助了不少人。我非常感谢 chelyaev (https://github.com/chelyaev/ffmpeg-tutorial) 他替换了我八年前写的程序里面过时的函数。

好,我希望这个教程是有益和有趣的。如果你有任何建议,问题,抱怨和赞美等,请给发邮件到dranger at gmail dot com。
